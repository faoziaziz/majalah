% Created 2020-10-06 Sel 16:45
% Intended LaTeX compiler: pdflatex
\documentclass[11pt]{article}
\usepackage[utf8]{inputenc}
\usepackage[T1]{fontenc}
\usepackage{graphicx}
\usepackage{grffile}
\usepackage{longtable}
\usepackage{wrapfig}
\usepackage{rotating}
\usepackage[normalem]{ulem}
\usepackage{amsmath}
\usepackage{textcomp}
\usepackage{amssymb}
\usepackage{capt-of}
\usepackage{hyperref}
\author{Aziz Amerul Faozi}
\date{\today}
\title{Developer C}
\hypersetup{
 pdfauthor={Aziz Amerul Faozi},
 pdftitle={Developer C},
 pdfkeywords={},
 pdfsubject={},
 pdfcreator={Emacs 26.3 (Org mode 9.1.9)}, 
 pdflang={English}}
\begin{document}

\maketitle
\tableofcontents


\section{Masih relevankah?}
\label{sec:org5e289c6}
Kita telah mengetahui bahwa sekarang banyak bahasa pemrogrman yang beredar di 
era sekarang python, C, java, dart\footnote{Bahasa tersebut biasanya sering dipakai pada aplikasi populer sekarang, 
seperti perangkat mobile, web, atau Machine Learning.}. Apakah bahasa C masih bisa diandalkan di
era sekarang?
\subsection{Framework}
\label{sec:org8dab84c}
Banyak framework dalam pyhton, java, dart misal django untuk development web API 
dengan python, java punya spring, javascript punya expressjs dan dart memiliki
flutter. Sebenarnya C juga memiliki framework yang bisa digunakan misal 
libmicrohttpd yang dibuat oleh GNU, atau mungkin nghttp2 yang bisa support 
pada protokol http2. Sayangnya atas nama kemudahan, dokumentasi, dan alasan
sebenarnya adalah marketing, framework bahasa C tidak lebih tenar dari framework
bahasa-bahasa yang populer saat ini. C dipandang kurang relevan pada kehidupan
developer sekarang, lebih banyak membuang waktu, dimana waktu adalah uang.
Serta kemudahan untuk developing yang masih dirasakan terlalu susah untuk 
memahami developing yang harus membaca api dan fungsi dari code dalam buatan C.
C memang imperative tidak support dengan paradigma OOP, walaupun tersedia struct
dalam bahasa C tapi tidak membuat itu cukup dengan kebutuhan sekarang. 
Orang lebih suka menggunakan OOP walau sebenarnya dia jarang menggunakan 
prosedure polimorfisme dan sebagainya, dan membangun sebuah software juga dalam 
paradigma bahasa C, atas nama pasar itulah saya tidak bisa menjamin bahwa 
bahasa C masih relevan dengan kehidupan kita sehari-hari.

\section{Kernel}
\label{sec:orgb17fca7}
Linux memang menjadi salah satu framework paling terkenal di driver. 
Menyatukan antara managemen konkurensi dengan driver hardware menjadi salah
satu andalan utama dari kernel linux. Linux dibangun dengan menggunakan bahasa C.
Memang perlu memahami alokasi memori dan pointer yang terkesan bertele-tele, tapi
tanpa itu mungkin seorang programmer tidak akan pernah secara bagus memahami
cara kerja prosesor yang dibangun ketika dia tidak mencobanya dengan bahasa C.

\section{Emebedded}
\label{sec:orgd81b1f7}
Lalu bagaimana dengan aplikasi lainnya? Kalau kita bisa membangun dengan mudah 
dengan menggunakan python atau javascript, mengapa kita masih harus menggunakan 
C? Sayangnya tidak semua aplikasi bagus untuk menggunakan python ataupun javascript,
salah satu yang bagus adalah aplikasi yang dibangun dengan C memang lebih 
tidak membutuhkan memory. Ada methode yang disebut cross-compile dan ini 
memberikan kita memprogram suatu perangkat dari perangkat lain tanpa harus 
menginstall dependency yang kita perlukan kedalam perangkat tersebut. 

Memang di era 90-an kita mengenal java aplikasi untuk perangkat mobile. Dan 
memory yang digunakan waktu itu memang sangat sedikit dibanding dengan smartphone
yang kita miliki sekarang, bukankah itu merupakan satu indikasi bahwa java
juga memiliki kesempatan yang sama dalam membangun perangkat embedded di 
era tahun 90-an itu?

\section{MCU atau MPU}
\label{sec:org2d62ebf}
Pilihan untuk menggunakan Microprosesor atau microcontroller juga merupakan 
suatu agenda dalam perancangan sistem Embedded, kita terkadang membutuhkan 
perangkat yang tidak memerlukan performa yang tinggi asalkan secara fisik 
kita menahan segala macam tantangan seperti ketahanan suhu atau aplikasinya harus
tetap berjalan dengan data yang dikirimkan sedikit. MCU menjadi pilihan utama
dalam kasus ini. Sayangnya terkadang juga perangkat Edge juga harus memiliki 
performa untuk memproses sinyal yang komplex seperti sinyal suara yang memiliki
kecerdasan buatan, dan pada layer session kita harus memprosesnya pada perangkat
edge tersebut, untuk itu kita memerlukan MPU. 

\section{Serevolusioner Einstein}
\label{sec:orgb978137}
\(E=mc^2\) menjadi salah satu produk unggulan dari Albert Einstein. Itu mungkin
berlaku di Fisika tapi bagaimana dengan bahasa Pemrogramman? Mungkin karena 
saking banyaknya revolusi yang terjadi sekarang bahasa C menjadi bahasa yang 
terpinggirkan.

\section{Emacs}
\label{sec:orge17e789}
Editor MACroS merupakan salah satu editor tertua selain vi dan ed. Emacs 
memang sangat unik dia bisa melalukan banyak hal, bukan hanya bisa digunakan
untuk membuka file dan mengeditnya tapi juga bisa digunakan sebagai alat untuk
membuka email, ada juga kecerdasan buata yang membuat anda bisa curhat dengan 
komputer. Dokumen ini pun saya buat dengan menggunakan Emacs.

\subsection{Kemampuan emacs}
\label{sec:org8bf28d9}
\begin{enumerate}
\item Membuat tabel dan digenerator kedalam pdf maupun html.
\end{enumerate}

\begin{center}
\begin{tabular}{lrr}
\hline
Grades & Mathematics & Physics\\
\hline
Ben & 9.2 & 9.9\\
Tom & 6.7 & 7.7\\
Tim & 7.5 & 6.7\\
Dean & 8.0 & 7.0\\
\end{tabular}
\end{center}

\section{GNU plot}
\label{sec:org0406881}
Plot merupakan sebuah cara untuk merepresentasikan data dalam bentuk visual,
code berikut adalah contoh penggunakan gnuplot dalam mengenerate plot 2D.
\begin{verbatim}
reset
set title "Check"
set xlabel "X"
set xrange [-8:8]
set xtics -8,2,8
set ylabel "Y"
set yrange [-20:70]
set ytics -20,10,70
f(x) = x**2
g(x) = x**3
h(x) = 10*sqrt(abs(x))
plot f(x) w lp lw 1, g(x) w p lw 2, h(x) w l lw 3
\end{verbatim}
Dan berikut adalah contoh plot dari code diatas
\subsection{library C}
\label{sec:orga88a1c9}
Membuat plot adalah salah satu kompetensi yang harus di kuasai oleh para
data visualisasi. GNU plot memberikan fasilitas itu. Selain itu gnu plot 
juga bisa digunakan dalam bahasa pemrograman C yang harus di manfaatkan 
dengan library GTK misalnya. 
\section{GTK3}
\label{sec:org06b42a2}
Untuk membuat sebuah aplikasi dashboard misalkan GTK menyediakan library untuk C.
Dengan gtk3 kita bisa membuat dashboard windows yang bisa kita manfaatkan sebagai
alat untuk menampilkan data.
\section{Masalah SSH}
\label{sec:orgc612a8c}
Suatu hari kita diajak bingung karena kita tidak bisa mengakses ssh melalui
tunnel SSH untuk itu kita membutuhkan teknik menambahkan pem kedalam 
sistem linux kita.
\begin{verbatim}
ssh-add aws.pem
ssh ubuntu@121.12.12.1
\end{verbatim}
misal kita memiliki akses keynya pem, maka untuk sekian kalinya anda tidak perlu
membutuhkan akses ssh lagi
\end{document}
